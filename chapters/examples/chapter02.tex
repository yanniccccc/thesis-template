\chapter{Micro-Frontend-Architekturen}
\label{ch:background}
Webanwendungen werden immer beliebter und erstetzen mehr und mehr die eigentlich funktionsreichen Desktopanwendungen. Mit dieser Beliebtheit kommt jedoch eine zunehmede Komplexität mit einher, die neue Herausforderungen hervorbringt. 

Micro Frontend Architekuren können in modernen Webanwendungen verwendet werden, um diese zunehmende Komplexität zu vereinfachen. Die Architektur suggeriert die Aufteilung einer monolythischen Anwendung in mehrere kleinere Anwendungen, welche dem End User wiederum nahtlos als gesamt Anwendung angezeigt werden. Durch solch eine Aufteilung können mehrere Entwickler Teams unabhängig voneinander an den kleineren Frontend Applikationen arbeiten und auch unabhängig voneinander testen und deployen \cite{10.1007/978-3-031-22792-9_8}.

Vertikale und Horizontale MFE






