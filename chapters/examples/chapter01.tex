\chapter{Einleitung}
\label{ch:intro}
In einer Welt, in der Webanwendungen zunehmend komplexer und anspruchsvoller werden, stehen Entwickler vor der Herausforderung, effiziente und skalierbare Architekturen zu implementieren. Micro-Frontend-Architekturen haben sich als vielversprechender Ansatz herauskristallisiert, um diese Komplexität zu bewältigen und die Entwicklung modularer, wartbarer Systeme zu ermöglichen. \cite{10.1145/3561846.3561853}

Die vorliegende Arbeit widmet sich der Analyse und dem Vergleich von Integrationsstrategien für Micro-Frontend-Architekturen in modernen Web-Applikationen. Im Fokus steht dabei die Fragestellung: "Welche Vor- und Nachteile bieten Web Components als Integrationsstrategie für Micro-Frontend-Architekturen im Vergleich zu alternativen Ansätzen in Bezug auf Entwicklungseffizienz, Wartbarkeit und Performance in modernen Web-Anwendungen?"


%
% Section: Motivation
%
\section{Motivation}
\label{sec:intro:motivation}
Die Motivation für diese Untersuchung ergibt sich aus der steigenden Komplexität moderner Webanwendungen, die neue Architekturansätze erfordert. Die Wahl der Integrationsstrategie hat direkte Auswirkungen auf die Performance der Webanwendung, was einen kritischen Faktor für Nutzererfahrung und Suchmaschinenoptimierung darstellt[6]. Trotz der Relevanz dieses Themas mangelt es an tiefgehenden Vergleichen der verschiedenen Integrationstechniken.
%
% Section: Ziele
%
\section{Ziel der Arbeit}
\label{sec:intro:goal}
Das Ziel dieser Arbeit ist es, durch eine gründliche Analyse der bestehenden Forschung und praktischen Erfahrungen zur Entscheidungsfindung für den am besten geeigneten Integrationsansatz beizutragen. Die These, die dabei untersucht wird, lautet: Web Components bieten als Integrationsstrategie für Micro-Frontend-Architekturen signifikante Vorteile in Bezug auf Standardisierung und Interoperabilität, was sie zu einer vielversprechenden Option im Vergleich zu alternativen Ansätzen wie iFrames und Module Federation macht.

%
% Section: Struktur der Arbeit
%
\section{Gliederung}
\label{sec:intro:structure}
Die Arbeit gliedert sich in mehrere Abschnitte: Nach dieser Einleitung werden zunächst die Grundlagen zu Micro-Frontend-Architekturen erläutert. Anschließend folgt eine detaillierte Betrachtung von Web Components als Integrationsstrategie, einschließlich ihrer technischen Grundlagen sowie Vor- und Nachteile. Ein umfassender Vergleich mit alternativen Ansätzen bildet den Kern der Untersuchung, wobei technische Unterschiede sowie Auswirkungen auf Entwicklungseffizienz, Wartbarkeit und Performance analysiert werden. Die Arbeit schließt mit einer Diskussion der Ergebnisse, einer Schlussfolgerung und einem Ausblick auf zukünftige Entwicklungen in diesem Bereich.

Durch die systematische Untersuchung und den Vergleich verschiedener Integrationsstrategien für Micro-Frontend-Architekturen soll diese Arbeit Entwicklern und Architekten wertvolle Einblicke und Entscheidungshilfen für die Gestaltung moderner, effizienter und wartbarer Webanwendungen liefern
